\documentclass[12pt, a4paper]{article}
\usepackage{listings}
\usepackage{paracol}
\usepackage[left=20mm, right=20mm, top=25mm, bottom=25mm]{geometry}
\usepackage{fancyhdr}
\usepackage{xcolor}

\usepackage{titling}
\setlength{\droptitle}{-2cm}

\usepackage{xeCJK}
\setCJKmainfont[AutoFakeBold=true, AutoFakeSlant=true]{DFKai-SB}
\setCJKmonofont[AutoFakeBold=true, AutoFakeSlant=true]{DFKai-SB}
\setCJKsansfont[AutoFakeBold=true, AutoFakeSlant=true]{DFKai-SB}

\usepackage{setspace}
\setstretch{1.1}

\pagestyle{fancy}
\fancyhf{}
\fancyhead[L]{\textbf{Rich Cake}} % Left header
\fancyhead[C]{\textit{By HyperSoWeak}} % Center header
\fancyhead[R]{\today} % Right header
\fancyfoot[C]{\thepage} % Center footer with page number

\setlength{\headheight}{15pt}

\lstset{
    frame=single, % Use a single frame around the code
    basicstyle=\ttfamily, % Typewriter font for code
    columns=flexible, % Flexible column width
    keepspaces=true, % Keep spaces in code
    breaklines=true, % Allow line breaking
    backgroundcolor=\color{white}, % Background color of the listing
    rulecolor=\color{black}, % Color of the frame
    xleftmargin=10pt, % Left padding
    xrightmargin=10pt, % Right padding
    aboveskip=5pt, % Space above the box
    belowskip=5pt, % Space below the box
    framexleftmargin=5pt, % Left margin inside frame
    framexrightmargin=5pt, % Right margin inside frame
    framexbottommargin=0pt, % Bottom margin inside frame
    framextopmargin=5pt % Top margin inside frame
}


\begin{document}

\begin{center}
    \parbox{0.8\textwidth}{
        \centering
        \vspace{0.5em}
        \LARGE \textbf{Rich Cake}\\[0.5em]
        \large{Time limit: 1s}\\
        \large{Memory limit: 256MB}\\[0.8em]
    }
\end{center}
\thispagestyle{fancy}

\section*{Problem Description}
你是一名頂尖的蛋糕師傅,專精於打造三層高的巨無霸生日蛋糕。每層蛋糕都有 $n$ 種不同的大小和價格,讓你可以靈活地組合出各種蛋糕。

不過,製作蛋糕可不是隨便疊的!根據蛋糕界不成文的物理法則,底層的蛋糕必須最大,中層要比底層小,頂層得比中層還小。要是你挑錯了大小,蛋糕就會直接崩塌成一堆奶油與慘叫!

今天,有一位超級暴發戶來到你的蛋糕店,提出了一個瘋狂的要求:「把你能做的所有蛋糕,每種都給我做一個!」作為蛋糕師傅,你立刻發現這是一個賺翻的機會。你的任務是計算出所有符合「底層最寬、中層較窄、頂層最小」規則的蛋糕組合,並且計算出這些蛋糕的總價格!

每個蛋糕的價格是其底層、中層和頂層蛋糕價格的總和,快來看看今天你能從這位暴發戶身上賺到多少吧!

\section*{Input Format}
輸入共七行:
\begin{itemize}
    \item 第一行是一個整數 $n$,表示每層蛋糕大小和價格的選項數量。
    \item 接下來兩行各有 $n$ 個整數,分別代表底層蛋糕的大小 $A_i$ 和價格 $P_i$。
    \item 接下來兩行各有 $n$ 個整數,分別代表中層蛋糕的大小 $B_i$ 和價格 $Q_i$。
    \item 接下來兩行各有 $n$ 個整數,分別代表頂層蛋糕的大小 $C_i$ 和價格 $R_i$。
\end{itemize}

\section*{Output Format}
輸出一個整數,代表所有合法蛋糕的總價格。

\section*{Constraints}
\begin{itemize}
    \item $1 \le n \le 10^5$
    \item $1 \le A_i, B_i, C_i \le 10^9$
    \item $1 \le P_i, Q_i, R_i \le 10^3$
\end{itemize}

\section*{Subtasks}
\begin{itemize}
    \item \textbf{Subtask 1 (5 points)}:$n = 1$
    \item \textbf{Subtask 2 (25 points)}:$n \le 300$
    \item \textbf{Subtask 3 (70 points)}:無其他限制
\end{itemize}

\begin{paracol}{2}
    \subsection*{Sample Input 1}
    \begin{lstlisting}
1
1
4
2
5
3
6        
    \end{lstlisting}
    \switchcolumn
    \subsection*{Sample Output 1}
    \begin{lstlisting}
15
    \end{lstlisting}
\end{paracol}

\begin{paracol}{2}
    \subsection*{Sample Input 2}
    \begin{lstlisting}
2
2 4
7 25
3 8
12 9
5 11
27 18        
    \end{lstlisting}
    \switchcolumn
    \subsection*{Sample Output 2}
    \begin{lstlisting}
169
    \end{lstlisting}
\end{paracol}

\end{document}
